\begin{anexosenv}

\partanexos

\chapter{Term Frequency-Inverse Document Frequency}

A técnica \textit{Term Frequency-Inverse Document Frequency (TFIDF)} é uma
medida estátistica usada prioritariamente para indicar a importância de um termo
em dado documento em relação a uma base de documentos \cite{1_rajaraman_ullman_2011}.
<<<<<<< HEAD

Para calcular tal fator, duas etapas distintas são necessárias:

\begin{itemize}
    \item \textbf{\textit{TF: Term Frequency: }} Calcula quantas vezes um termo
    aparece em um dado documento. Dessa forma, seu cálculo pode realizado pela
    seguinte fórmula:

    TF(t) = N(t,d)/T(d)

    Onde: 
    \begin{itemize}
        \item \textbf{N(t, d): } Número de vezes que o termo ``t'' aparece no
        documento ``d''.
        \item \textbf{T(D): } Número de termos presentes no documento ``d''.
    \end{itemize}

    \item \textbf{\textit{IDF: Inverse Document Frequency: }} Usada para diminuir o
    peso de termos muito frequentes em documentos, como os termos ``para'' ou
    ``é''. Para fazer isso, a seguinte fórmula pode ser usada:

    IDF(t) = log(NumDocumentos/D(t))

    Onde:
    \begin{itemize}
        \item \textbf{NumDocumentos: } Quantidade de documentos presentes na
        base.
        \item \textbf{D(T): } Número de documentos que contém o termo ``t''.
    \end{itemize}
\end{itemize}

Com os fatore \textbf{TF} e \textbf{IDF} calculados, basta multiplicar tais
valores para encontar o valor da importância de um termo dado uma base de
documentos. Sendo assim, pode-se entender que o valor do \textit{TFIDF} será
alto para um termo ``t'' se o mesmo é encontrado muitas vezes em um dado
documento e poucas vezes na coleção de documentos.

Existem também outras formas de calcular os valores \textit{TF} e \textit{IDF}.
Essas formas tendem a priorizar certos aspectos em relação ao texto. Uma dessas
abordagens é chamada de \textit{TFIDF-sublinear}. Essa técnica é usada para
atenuar o valor do \textit{TF} para quando um termo é bastante presente em um
documento, mas não é muito presente na base de documentos. A atenuação do valor
do \textit{TF} pode se justificar em casos quando um termo que aparece mais de
uma vez no documento não deva ter um peso substancialmente maior que um termo
que aparece uma única vez. Para calcular o \textit{TF} para este caso, pode-se
usar a seguinte fórmula:

TF-sublinear(t) = 1 + log(TF(t))

Vale ressaltar que outras abordagens também podem ser usadas, como normalização
usando o tamanho médio dos documentos presentes na base de todos os documentos e
\textit{TF} médio entre todos os \textit{TF}s \cite{araujo2011apprecommender}.
=======
>>>>>>> Adicionando anexos

Para calcular tal fator, duas etapas distintas são necessárias:

\begin{itemize}
    \item \textbf{\textit{TF: Term Frequency: }} Calcula quantas vezes um termo
    aparece em um dado documento. Dessa forma, seu cálculo pode realizado pela
    seguinte fórmula:

    TF(t) = N(t,d)/T(d)

    Onde: 
    \begin{itemize}
        \item \textbf{N(t, d): } Número de vezes que o termo ``t'' aparece no
        documento ``d''.
        \item \textbf{T(D): } Número de termos presentes no documento ``d''.
    \end{itemize}

    \item \textbf{\textit{IDF: Inverse Document Frequency: }} Usada para diminuir o
    peso de termos muito frequentes em documentos, como os termos ``para'' ou
    ``é''. Para fazer isso, a seguinte fórmula pode ser usada:

    IDF(t) = log(NumDocumentos/D(t))

    Onde:
    \begin{itemize}
        \item \textbf{NumDocumentos: } Quantidade de documentos presentes na
        base.
        \item \textbf{D(T): } Número de documentos que contém o termo ``t''.
    \end{itemize}
\end{itemize}

Com os fatore \textbf{TF} e \textbf{IDF} calculados, basta multiplicar tais
valores para encontar o valor da importância de um termo dado uma base de
documentos. Sendo assim, pode-se entender que o valor do \textit{TFIDF} será
alto para um termo ``t'' se o mesmo é encontrado muitas vezes em um dado
documento e poucas vezes na coleção de documentos.

Existem também outras formas de calcular os valores \textit{TF} e \textit{IDF}.
Essas formas tendem a priorizar certos aspectos em relação ao texto. Uma dessas
abordagens é chamada de \textit{TFIDF-sublinear}. Essa técnica é usada para
atenuar o valor do \textit{TF} para quando um termo é bastante presente em um
documento, mas não é muito presente na base de documentos. A atenuação do valor
do \textit{TF} pode se justificar em casos quando um termo que aparece mais de
uma vez no documento não deva ter um peso substancialmente maior que um termo
que aparece uma única vez. Para calcular o \textit{TF} para este caso, pode-se
usar a seguinte fórmula:

TF-sublinear(t) = 1 + log(TF(t))

Vale ressaltar que outras abordagens também podem ser usadas, como normalização
usando o tamanho médio dos documentos presentes na base de todos os documentos e
\textit{TF} médio entre todos os \textit{TF}s \cite{araujo2011apprecommender}.

\chapter{Termo de consentimento livre e esclarecido}

O (a) Senhor(a) está sendo convidado(a) a participar da pesquisa
\textit{Estratégia de contexto temporal em recomendação de pacotes GNU/Linux}
 
Este pesquisa tem como objetivo coletar os seguintes dados do usuário:

\begin{itemize}
    \item Versão do sistema operacional do usuário;
    \item Versão do kernel do usuário;
    \item Versão do processador do usuário;
    \item Lista de todos os pacotes instalados na máquina do usuário;
    \item Lista de todos os pacotes manualmente instalados;
    \item Tempo de acesso e modificação de todos os pacotes manualmente instalados;
    \item Caminho de todos os pacotes manualmente instalados;
    \item Submissão do usuário para o sistema popularity-contest;
    \item Data e hora da coleta
\end{itemize}

O(a) senhor(a) receberá todos os esclarecimentos necessários antes e no decorrer da pesquisa e
lhe asseguramos que seu nome não aparecerá, sendo mantido o mais rigoroso sigilo através da omissão total de quaisquer
informações que permitam identificá-lo(a)

A sua participação será através da instação da aplicação AppRecommender na sua própria máquina,
juntamente com a execução de um script responsável pela coleta dos dados citados como objetivo da pesquisa.
Sua participação é voluntária, isto é, não há pagamento por sua colaboração.

Os resultados da pesquisa serão divulgados na primeira parte do trabalho de conclusão de curso
dos alunos Lucas Albuquerque Medeiros de Moura e Luciano Prestes Calvancati. 

Se o(a) Senhor(a) tiver qualquer dúvida em relação à pesquisa, por 
favor mande um e-mail para: lucas.moura128@gmail.com ou lucianopcbr@gmail.com.

Este documento foi elaborado em duas vias, uma ficará com o pesquisador responsável e
a outra com o sujeito da pesquisa.

Ao concordar com este termo, o senhor(a) cederá os dados coletados sem recompensa financeira e 
para uso em pesquisa científica sem fins lucrativos apenas. Esses dados não serão repassados para terceiros.

\begin{center}
\begin{tabular}{ll}
\centerline{\makebox[2.5in]{\hrulefill}}\\
\centerline{Nome/Assinatura}\\[8ex]% adds space between the two sets of signatures
\centerline{\makebox[2.5in]{\hrulefill}}\\
\centerline{Pesquisador Responsável}\\[8ex]% adds space between the two sets of signatures
\end{tabular}
\end{center}

\hfill Brasília, \makebox[0.5in]{\hrulefill} de \makebox[1.5in]{\hrulefill} de \makebox[1in]{\hrulefill} 

\end{anexosenv}

