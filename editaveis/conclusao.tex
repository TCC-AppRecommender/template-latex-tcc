\chapter[Conclusão]{Conclusão}

Com os resultados obtidos, pode-se perceber que ambas as hipóteses, \textit{h1}
e \textit{h2} não podem ser negadas. Para hipótese
\textit{h1}, que se trata da recomendação ser mais apropriada ao aumentar o
peso dos pacotes recentemente usados no sistema, percebeu-se um aumento em todas
as estratégias de contexto temporal quanto a precisão das recomendações,
entretanto, a estratégia determinística também teve um aumento de recomendações
negativas ao usuário, o que leva-se a entender que o contexto temporal sozinho
não foi responsável pelas melhorias. Para a hipótese \textit{h2}, que dita que
a estratégia de aprendizado de máquina obtém melhores resultados que a estratégia
determinística, percebeu-se que sim, as estratégias de aprendizado de máquina se
comportaram melhor que a determinística. Acredita-se que esse foi
o caso devido a combinação de pré-filtragem dos pacotes juntamente com o contexto
temporal que acontecia nas estratégias de aprendizado de máquina.

Com a melhoria encontrada nos resultados obtidos, pode-se entender também a
importância de uma melhor filtragem do perfil de usuário antes da recomendação
em si. Sendo assim, foi entendido que o contexto temporal proposto não precisa
estar associado apenas a estratégia de recomendação em si, mas também a todas as
etapas da recomendação. Isso se mostrou pertinente neste trabalho, pois o
contexto temporal foi aplicado tanto na pós-filtragem dos pacotes, quanto na
composiçãod dos pacotes no perfil do usuário.

Com esta análise dos resultados, entendeu-se também que o foco da
recomendação não deve estar apenas em qual estratégia utilizar para fazer a
recomendação, e sim na seleção de quais pacotes devem alimentar o recomendador e
depois usar uma estratégia para representar e selecionar os pacotes para uma
recomendação.

Além disso, a pesquisa também permitiu entender que algumas
contribuições podem ser feitas ao próprio sistema Debian para melhorar ainda
mais os resultados de uma recomendação. A principal delas seria a criação de um log para o uso de
pacotes, similar ao que acontece com o \textit{relatime}, onde diariamente, seria registrado
quais pacotes tiveram seu \textit{atime} alterado. Esse arquivo iria conter a frequência
de uso dos pacotes pelo usuário, provendo essa frequência de uso dos pacotes ao
contexto temporal, permitindo uma análise ainda mais próxima do perfil do usuário.

Ainda se tratando dos resultados encontrados, verificando a média de acurácia do
\textit{cross validation} e o número de dados médio para o treinamento dos
algoritmos, pode-se entender que uma abordagem de aprendizado de máquina que
consiga resultados mais expressivos de acurácia e que se comporte melhor para
menores conjuntos de dados, pode melhorar ainda mais os resultados encontrados.

Por fim, vale ressaltar que este
trabalho possibilitou que ambos os pesquisadores envolvidos trabalhassem no
programa \textit{Google Summer of Code}, onde a instituição \textit{Google}
financia estudantes para trabalharem com software livre durante o verão. Os
pesquisadores deste trabalho estão trabalhando também com o
\textit{AppRecommender}, porém em outras áreas não relacionadas a esta pesquisa.
Dessa forma, pode-se dizer que além dos resultados obtidos, esta pesquisa foi
capaz de reviver o \textit{AppRecommender} dentro do projeto \textit{Debian},
e também pretende atrair mais pesquisadores para dar continuidade ao projeto.

\section{Limitações do trabalho}

Uma das principais limitações deste trabalho se encontra no aspecto subjetivo de
uma avaliação de uma recomendação e no grupo de usuários que realizaram a coleta
de opinião. Considerando que apenas graduados e graduandos do curso de Engenharia
de Software da Universidade de Brasília foram consultados, pode existir talvez
um perfil comum entre os participantes no que tange os resultados encontrados.

De forma a melhor verificar os resultados encontrados nesta pesquisa, decidiu-se
por levantar a hipótese de realizar uma coleta de opinião em outros dois eventos
de software livre, a \textit{DebConf}, conferência anual do projeto Debian, e o
Fórum Internacional De Software Livre (FISL), evento anual de encontro de
comunidades de software livre. Com as consultas realizadas nesses dois eventos,
pretende-se então construir um perfil de usuário mais heterogêneo, permitindo
assim uma melhor análise das estratégias desenvolvidas.

\section{Trabalhos futuros}

Dado as questões levantadas pela pesquisa, foi possível levantar alguns
possíveis trabalhos futuros:

\begin{itemize}
  \item \textbf{Uso do arquivo history do usuário para complementar contexto
  temporal:} Apesar do arquivo \textit{bash\_history} do usuário apenas conter um
  histórico de comandos usados pelo usuário, tal arquivo pode conter informações
  importantes sobre a frequência de uso de algumas aplicações pelo usuário,
  podendo complementar a análise temporal feita.
  \item \textbf{Melhorar estratégias de aprendizado de máquina:} Dado os
  problemas já citados pelas estratégias de aprendizado de máquina, pode-se
  investigar novas metodologias para lidar com os problemas citados, como poucos
  dados de treinamento e complexidade de uso dos mesmos.
  \item \textbf{\textit{Cluster} de interesse do usuário:} Essa nova estratégia seria
  para criar \textit{clusters} de gostos do usuário com base em seus pacotes e
  buscar pacotes baseado na identidade desse \textit{cluster} e não nos pacotes
  em si.
  \item \textbf{Melhorar contexto temporal:} As estratégias de contexto
  temporal ainda apresentam problemas para pacotes recentemente atualizados.
  Isso se dá pelo fato que ainda não é possível saber se um pacote foi
  recentemente utilizado e atualizado pelo sistema, pois a operação de
  atualização opera tanto sobre o \textit{ctime} quanto \textit{atime}. Dessa
  forma, pode-se buscar maneira de resolver tal problema.
  \item \textbf{Evitar redundância de recomendação de pacotes:} Pacotes
  instalados pelo usuário por outros gerenciadores de pacotes, como
  \textit{rvm} do ruby ou o \textit{pip} do python, não devem ser
  recomendados. Dessa forma, deve-se entender um mecanismo de verificar a
  existência desses pacotes e retirá-los da recomendação. Além disso, deve-se
  verificar também redundância entre pacotes dentro de uma recomendação,
  pois dois pacotes similares não devem estar presentes na mesma
  recomendação.
  \item \textbf{Uso de experimentos \textit{offline}:} Experimentos \textit{offline} eram
  implementados no \textit{AppRecommender} com o intuito de comparar as
  estratégias de forma mais rápida e prática, pois não era necessário
  sempre fazer testes com usuários. Para o contexto temporal usado, tais
  experimentos deveriam ser modificados e melhor explorados para se
  adequar a finalidade da pesquisa.
\end{itemize}
