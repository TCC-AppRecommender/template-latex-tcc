\chapter[Conclusão]{Conclusão}

Com os resultados obtidos, pode-se perceber que ambas as hipóteses, \textit{h1}
e \textit{h2} se provaram verdadeiras, porém com ressalvas. Para hipótese
\textit{h1}, percebeu-se um aumento em todas as estratégias de contexto
temporal quanto a precisão das recomendações, entretanto, a estratégia
determinística também teve um aumento de recomendações negativas ao usuário, o
que dessa forma, leva-se a entender que o contexto temporal sozinho não foi
responsável pelas melhorias. Para a hipótese \textit{h2}, percebeu-se que sim,
as estratégias de aprendizado de máquina se comportaram melhor que a
determinística, mas novamente, acredita-se que esse foi o caso pela combinação
de pré-filtragem dos pacotes juntamente com o contexto temporal que acontecia
nas estratégias de aprendizado de máquina, o que não era o caso da estratégia
determinística, que só usava o contexto temporal e não o de pré-filtragem.

Dessa forma, os responsáveis por esta pesquisa entendem que para recomendações
baseadas em conteúdo, o perfil do usuário teve relação bem mais importante com o
sucesso das recomendações do que as estratégias propriamente ditas. Percebeu-se
que as estratégias de recomendação melhoram quando usou-se o contexto temporal
para melhor filtrar os pacotes usados para compor o perfil do usuário e não para
selecionar os pacotes após a recomendação.

Dessa forma, acredita-se que para a recomendação de pacotes, esta pesquisa
entende que o foco da recomendação não deve estar apenas nas estratégias usadas
para fazer a recomendação propriamente dita e sim no conjunto de seleção de
quais pacotes devem alimentar o recomendador e depois usar uma estratégia
para representar e selecionar os pacotes para uma recomendação

Além disso, os responsáveis pela pesquisa entendem também que existem algumas
contribuições que podem ser feitas ao próprio sistema Debian que possa facilitar
a tarefa de recomendação. A principal dela seria a criação de um log para uso de
pacotes similar ao que acontece com o relatime. Diariamente, seria registrado
quais pacotes tiveram seu atime alterado. Esse arquivo seria então a frequência
de uso de pacote pelo usuário, o que proveria ao contexto temporal a frequência
de uso de pacotes, fazendo assim uma análise ainda mais próxima ao perfil do
usuário.

Conclui-se então que a pesquisa feita usou um bom indicador para melhorar as
recomendações de pacotes, o contexto temporal de seu uso, entretanto,
percebeu-se que a pesquisa deveria ter focado mais no conjunto de dados usado
para alimentar o sistema de recomendação e também no relacionamento entre tais
pacotes desenvolvidos, pois apesar da melhoria observada pela novas estratégias,
percebeu-se pelo segundo experimento uma relação mais direta entre os dados que
são usados para alimentar o recomendador e a recomendação propriamente dita.

Além disso, pelos resultados encontrados pela média de acurácia do \textit{cross
validation} e o número de dados médio para o treinamento dos algoritmos, pode-se
entender também que abordagens mais tradicionais de aprendizado de máquina não
são suficientes para o contexto do sistema de recomendação usado.

Por fim, vale ressaltar que mesmo com os resultados não tão expressivos, este
trabalho possibilitou que ambos os pesquisadores envolvidos trabalhassem no
programa \textit{Google Summer of Code}, onde a instituição \textit{Google}
financia estudantes para trabalharem com software livre durante o verão. Os
pesquisadores deste trabalho estão trabalhando também com o
\textit{AppRecommender}, porém em outras áreas não relacionadas a esta pesquisa.
Dessa forma, pode-se dizer que além dos resultados obtidos, esta pesquisa foi
capaz de reviver o \textit{AppRecommender} dentro do projeto \textit{Debian},
e também pretende atrair mais pesquisadores para dar continuidade ao projeto.
