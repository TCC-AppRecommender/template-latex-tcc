\chapter[Considerações preliminares]{Considerações preliminares}

No decorrer deste trabalho, alguns marcos foram alcançados. Primeiramente, os
integrantes da pesquisa estabeleceram um bom conhecimento sobre sistemas de
recomendação e aprendizado de máquina, fazendo então possível que esses
conhecimentos sejam colocados em prática.

Além disso, modificações foram realizadas no projeto escolhido para ser usado
nesta pesquisa, o AppRecommender. Primeiramente, as dependências de pacote do
projeto foram atualizadas e um estudo foi realizado com o intuito de prover os
meios necessários para o projeto funcionar. Esse objetivo foi alcançado e
algumas técnicas de engenharia de software foram também aplicadas para tornar o
desenvolvimento do projeto mais fácil, como atualizar os testes existentes, adicionar novos testes
e centralizar a execução dos mesmos além prover também uma forma automatizada de
verificar o padrão de codificação.

Para os objetivos da pesquisa, já está implementada uma forma de coletar e
classificar os pacotes com bases em suas informações temporais, além da criação
da estratégia de recomendação utilizando a abordagem determinística proposta no
trabalho, entretanto, os pesos usados na mesma ainda precisam ser melhor escolhidos.

Para a estratégia de aprendizado de máquina, os scripts necessários para coletar
os dados dos pacotes, extrair seus atributos e classificá-los também estão
prontos. A técnica proposta para realizar o aprendizado de máquina também está
implementada, porém tal implementação ainda carece de algumas melhorias,
principalmente na parte de teste.

Por fim, o script para uma coleta de dados foi finalizado também, faltando
apenas a parte gráfica do mesmo, para prover a interação com o usuário.

Baseado nas considerações feitas será apresentado na seção \ref{sec:cronagrama} um cronograma
para a continuação da pesquisa.

\section{Cronograma} \label{sec:cronagrama}

Afim de apresentar uma continuação para este trabalho, a tabela~\ref{tab:atividades_futuras},
apresenta as atividades a serem realizadas e suas respectivas datas.

\begin{table}[h]
\centering
\resizebox{\textwidth}{!}{\begin{tabular}{|l|l|l|}
\hline
\rowcolor[HTML]{EFEFEF}
{\textbf{Atividade}} & {\textbf{Data de Início}} & {\textbf{Data de Término}} \\ \hline
Finalizar aprendizado de máquina & 01/01/2016 & 15/02/2016 \\
Interface da coleta & 16/02/2016 & 29/02/2016 \\
Primeira coleta / análise dos resultados & 01/03/2016 & 31/03/2016 \\
Formalizar resultado da coleta & 01/04/2016 & 30/04/2016 \\
Segunda coleta / teste de hipótese & 01/05/2016 & 31/05/2016 \\
Finalizar documento & 01/06/2016 & 30/06/2016 \\
\hline
\end{tabular}}
\caption{Atividades a serem realizadas}
\label{tab:atividades_futuras}
\end{table}

O objetivo inicial na próxima fase do projeto é finalizar o aprendizado
de máquina. Onde o primeiro passo a fazer é a implementação do aprendizado
de máquina, dos testes, e também implementar tanto a validação cruzada quanto
curvas de aprendizado, possibilitando assim avaliar o modelo desenvolvido.

Após o aprendizado de máquina estar implementado este deve ser adicionado
em uma estratégia de recomendação, ou seja, será criado uma estratégia
no AppRecommender que utiliza o aprendizado de máquina para recomendar
pacotes.

A coleta de dados do usuário já está implementada, porém será adicionado
uma interface gráfica, visando otimizar a interação do usuário com o
software. Após a implementação da interface gráfica será realizado a
coleta de dados dos usuários, e utilizando esses dados será realizado
uma análise a fim de otimizar a estratégia determinística e também
implementar a análise de curva ROC.

Após essa primeira coleta de dados e da análise dos resultados, será
estudado a apresentação dos dados, visualização dos algoritmos e também
verificar se os algoritmos precisam ser adaptados como por exemplo
diminuir o número de atributos usados na técnica de aprendizado de máquina.

Então os resultados da coleta serão formalizados, afim de que o leitor
possa compreender para que servem os dados e o que foi realizado com esses
dados para se chegar aos resultados. Utilizando esses resultados também
será feita uma atualização nas fórmulas e nas estratégias de recomendações
criadas, afim de otimizar a recomendação.

Depois de se compreender os dados, os resultados e otimizar a recomendação
será feita uma segunda coleta de dados, onde com os resultados dessa coleta
será realizado o teste de hipótese será enfim realizado.

Por último, os resultado serão estão descritos na segunda etapa da pequisa e o
texto será revisado e atualizado mediante os resultados encontrados
