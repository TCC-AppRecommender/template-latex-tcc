\chapter[Introdução]{Introdução}

Segundo \citeonline{burke2002hybrid}, sistemas de recomendação são aqueles que
produzem recomendações individualizadas como saída ou que guiam o usuário
de forma personalizada para itens interessantes ou úteis em um conjunto
grande de opções. Com essa definição em mente, muitas empresas buscam
impulsionar tanto a venda quanto a visualização dos itens que possuem, e até mesmo
melhor direcionar produtos para o usuário. Exemplos clássicos desse uso são a
\textit{Amazon} provendo um loja online praticamente toda personalizada
\cite{jannach2008finding}, e a empresa \textit{Netflix} que usa sistemas de recomendação
para prover filmes a seus usuários. Vale ressaltar que esta empresa recentemente
realizou um concurso no valor de um milhão de doláres para quem pudesse melhorar
a performance do seu sistema de recomendação \cite{koren2009matrix}.

Com essa informação em mente, percebe-se que sistemas de recomendação podem
trazer benefícios para grande gama de aplicações em contextos diferentes. Um
desses contextos pode ser a recomendação de pacotes GNU/Linux. Um dos sistemas
GNU/Linux mais popular é o Debian, que no período de realização deste trabalho
possui 49.096 pacotes em seus repositórios \footnote{Dados coletados
08/11/2015}. Dado esse número elevado de pacotes, usuários podem ter
dificuldades em encontrar pacotes que lhes agradem ou até mesmo ficarem sabendo
de algum novo pacote no sistema. Além disso, sistemas de recomendação também
podem ajudar a comunidade, pois com mais usuários do pacote, pode-se entender que
o número de \textit{bugs} registrados tende a aumentar, podendo assim aumentar a qualidade
do software sendo usado.

Baseado neste contexto, um sistema de recomendação de pacotes, chamado
\textit{AppRecommender} foi desenvolvido para sistemas Debian pela
\textit{Debian Developer} Tássia Camões Araújo durante
o seu mestrado na Universidade de São Paulo no ano de 2011. Este sistema visa a
recomendação de pacotes baseado em quesitos como os pacotes já instalados pelo
usuário ou até mesmo submissões do \textit{popularity-contest}. Dessa forma, o
projeto abrange desde recomendações por conteúdo a até mesmo recomendações
colaborativas.

Entretanto, dado a gama de pacotes disponíveis em sistemas Debian, prover uma
recomendação útil para o usuário pode se tornar um problema. O problema de
utilidade das recomendações também se prova presente no \textit{AppRecommender},
onde o resultado das recomendações por conteúdo não se mostrou tão satisfatória
\cite{araujo2011apprecommender}. Dessa forma, partindo do
pressuposto de que o usuário usa apenas uma parte dos seus pacotes em um dado
momento, entender o contexto temporal do uso de pacotes pode acarretar em recomendações reais e
mais úteis ao usuário.

Afim de colaborar com o sistema Debian, esta pesquisa visa juntar informações
temporais de uso de pacotes a um sistema de recomendação de pacotes já existente.
Para isso, duas abordagens serão propostas, uma determinística e uma por aprendizado
de máquina. Com essas abordagens, a pesquisa visa então entender se o contexto temporal
irá trazer melhorias para as recomendações feitas pelo sistema.

\section{Justificativa}

Sabendo do crescente número de pacotes presentes no repositório do Debian e a
necessidade de se criar uma comunidade em torno desses pacotes, este trabalho
visa ajudar tantos os usuários Debian que gostariam de usar mais pacotes que
lhes forneçam funcionalidades interessantes no seu dia-a-dia, quanto a própria
comunidade Debian, ao possibilitar que mais usuários usem determinados pacotes,
podendo assim aumentar a comunidade em torno do mesmo.

Além disso, esta pesquisa também visa servir de insumo para outros estudos
relacionados a recomendação de pacotes GNU/Linux ou até mesmo pesquisas
interessadas em aplicar contexto temporal em seus sistemas de recomendação.

\section{Objetivo}

Este trabalho tem como principal objetivo estudar como o contexto temporal pode
afetar a recomendação de pacotes GNU/Linux. Nesta etapa inicial do projeto,
pretende-se especificar os modelos que serão usados para extrair informação
temporal de um pacote, além de modelos de uso dessas informações em um sistema
de recomendação.

Para a segunda parte desta pesquisa, o objetivo será a comparação direta dos
modelos descritos via um consulta de opinião de usuário quanto a efetividade
das recomendações produzidas.

Dessa forma, pode-se dizer que está pesquisa visa responder a seguinte questão-problema:

\textit{"É possível melhorar a recomendação de pacotes usando contexto temporal
dos pacotes de um usuário"}

\section{Organização do trabalho}

Este trabalho está dividido em 3 capítulos distintos. No Capítulo 2,
são apresentadas as revisões bibliográficas feitas para sistemas de recomendação e aprendizado de máquina.
No Capítulo 3, é apresentada a metodologia proposta para esta pesquisa, contendo desde as hipóteses levantadas para a pesquisa,
 como o sistema de recomendação usado como base e como os resultados obtidos serão comparados. Por fim, o último capítulo é usado para relatar
algumas considerações do trabalho sendo desenvolvido, como os resultados obtidos até o momento, além de atividades que ainda precisam ser feitas,
usando um contexto de um cronograma para relatar tais atividades.
