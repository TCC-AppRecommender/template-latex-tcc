\chapter*[Introdução]{Introdução}
\addcontentsline{toc}{chapter}{Introdução}

Segundo \citeonline{burke2002hybrid}, sistemas de recomendação são aqueles que
produzem recomendações individualizadas como saída ou que guiam o usuário
usuário de forma personalizada para itens interessantes ou úteis em um conjunto
grande de opções. Com essa definição em mente, muitas empresas buscam
impulsionar tanto a venda como a visualização dos itens que possuem a até mesmo
melhor direcionar produtos para o usuário. Exemplo clássicos desse uso são a
Amazon provendo um loja online praticamente toda personalizada
\cite{jannach2008finding}, e a empresa Netflix que usa sistema de recomendação
para prover filmes a seus usuários. Vale ressaltar que esta empresa recentemente
realizou um concurso no valor de um milhão de doláres para quem pudesse melhorar
a performance do seu sistema de recomendação \cite{koren2009matrix}.

Com essa informação em mente, pode-se ver que sistemas de recomendação podem
trazer benefícios para grande gama de aplicações em contextos diferentes. Um
desses contextos pode ser a recomendação de pacotes GNU/Linux. Um dos sistemas
GNU/Linux que existem é o Debian, que no período de realização desse trabalho
possui 49096 pacotes para em seus repositórios \footnote{Dados coletados
08/11/2015}. Dado este número elevado de pacotes, usuários podem ter
dificuldades em encontrar pacotes que lhes agradem ou até mesmo ficarem sabendo
de algum novo pacote no sistema. Além disso, sistemas de recomendação também
podem ajudar a comunidade, pois com mais usuários do pacote, pode-se entender que
o número de \textit{bugs} registrados possa aumentar, podendo assim aumentar a qualidade
do software sendo usado.

Entretanto, dado a gama de pacotes disponíveis em sistemas Debian, prover uma
recomendação útil para o usuário pode se tornar um problema. Partindo do
pressuposto que o usuário usa apenas uma parte dos seus pacotes em um dado
momento, entender o contexto temporal  do uso de pacotes pode acarretar em recomendações reais e
mais úteis ao usuário.

Afim de colaborar com o sistema Debian esta pesquisa visa juntar informações
temporais de uso de pacotes à um sistema de recomendação de pacotes já existente.
Para isso duas abordagens serão propostas, uma determinística e uma por aprendizado
de máquina. Com essas abordagens, a pesquisa visa estão entender se o contexto temporal
irá trazer melhorias para as recomendações feitas pelo sistema.

\section{Justificativa}

Sabendo do crescente número de pacotes presentes no repositório do Debian e a
necessidade de se criar uma comunidade em torno desses pacotes, este trabalho
visa ajudar tantos os usuários Debian que gostariam de usar mais pacotes que
lhes forneçam funcionalidades interessantes no seu dia-a-dia, quanto a própria
comunidade Debian, ao possibilitar que mais usuários usem determinados pacotes,
podendo assim aumentar a comunidade em torno do mesmo.

Além disso, esta pesquisa também visa servir de insumo para outros estudos
relacionados a recomendação de pacotes GNU/Linux ou até mesmo pesquisas
interessadas em aplicar contexto temporal em seus sistemas de recomendação.

\section{Objetivo}

Este trabalho tem como principal objetivo estudar como contexto temporal pode
afetar a recomendação de pacotes GNU/Linux. Nesta etapa inicial do projeto,
pretende-se especificar os modelos que serão usados para extrair informação
temporal de um pacote, além de modelos de uso dessas informações em um sistema
de recomendação.

No entanto, para realizar tais feitos, será necessário um aprofundamento na
teoria de sistemas de recomendação e aprendizado de máquina, com o intuinto de
dominar tais conceitos, principalmente as peculiaridades de cada um.

Com o conhecimento teórico adquirido, a parte prática do projeto pode começar a
ser realizada, onde um software livre de recomendação de pacotes será usado como
base para implementação das novas abordagens contendo informação temporal. Além
disso, será necessário também realizar comparações entre as abordagens originais
propostas pela aplicação e as novas, propostas pela pesquisa.

Com a comparação realizada, pode-se então analisar o impacto do contexto
temporal nas recomendações realizadas, visando principalmente averiguar se as
recomendações estão consoantes com as expectativas do usuário.

Em resumo, está pesquisa visa responder a seguinte questão-problema:

\textit{"É possível melhorar a recomendação de pacotes usando contexto temporal
dos pacotes de um usuário ?"}

\section{Organização do trabalho}

Este trabalho está dividido em 3 capítulos distintos. No capítulo 2,
são apresentadas as revisões bibliográficas feitas para sistemas de recomendação e aprendizado de máquina.
No capítulo 3, é apresentado a metodologia proposta para está pesquisa, contendo desde as hipóteses levantadas para a pesquisa,
 como o sistema de recomendação usado como base e como os resultados obtidos serão comparados. Por fim, o último capítulo é usado para relatar
algumas considerações do trabalho sendo desenvolvido, como os resultados obtidos até o momento, além de atividades que ainda precisam ser feitas,
usando um contexto de um cronograma para relatar tais atividades.
