\part{Metodologia}

\chapter[Metodologia]{Metodologia}

Este capítulo é destinado à apresentar toda a metodologia aplicada a esta
pesquisa. Vale ressaltar que a metodologia proposta nesta pesquisa está sendo
baseada na definição proposta por Silva e Menezes, que afirmam que toda pesquisa
tem como objetivo encontrar respostas para hipóteses propostas
\cite{da2005metodologia}

Com essa definição em pauta, as hipóteses proposta para a execução deste trabalho neste trabalho foi a seguinte:

\begin{center}
\textit{É possível melhorar a recomendação baseada em conteúdo de pacotes para usuários de sistemas GNU/Linux levando em consideração o tempo de uso dos pacotes já instalados ?}
\end{center}

\section{Hipóteses}

Considerando um sistema de recomendação baseado em conteúdo, sendo que os dados principais de entrada para a recomendação de outros pacotes são os pacotes já instalados pelos usuários, as seguinte hipóteses foram definidas:

\begin{itemize}
    \item \textit{\textbf{Ao aumentar o peso de pacotes recentemente usados no sistema, o perfil do usuário gerado será mais preciso, ocasionando assim uma recomendação mais apropriada.}} Considerando que o perfil do usuário na recomendação é gerado pelo termos encontrados em maior presença em todos os pacotes manualmente instalados pelo usuário, é uma possibilidade aumentar o peso dos termos de pacotes recentemente usados e diminuir o peso de pacotes antigos, irá gerar um perfil que melhor espelha as prioridades do usuário, gerando assim uma recomendação de maior precisão
    \item \textit{\textbf{Uma abordagem determinística para o problema irá se comportar pior que uma não-determinística quando se usa o tempo de uso como critério de classificação}} Uma abordagem deterministica iria modificar o atual perfil do usuário usado para gerar a recomendação sendo gerada, entretanto uma abordagem não deterministica, baseada no aprendizado de máquina, irá refinar as recomendações geradas usando a configuração atual. Considerando a complexidade dos dados presentes em um pacote Debian, juntamente com os relacionamentos entre tais atributos, acredita-se que uma abordagem não-deterministica irá prover melhores resultados.
\end{itemize}

\section{Planejamento da pesquisa}

Com o intuito de responder a questão principal do trabalho, juntamente com as hipóteses propostas, um fluxo de trabalho foi definido. A Figura x sistematiza tal fluxo, facilitando assim a visualização do processo adotado.

\begin{enumerate}
    \item \textit{\textbf{Selecionar o projeto}}
Selecionar um projeto de software livre que implemente um sistema de recomendação baseada em conteúdo e que não leve em consideração questões temporais do pacote quando se realiza tanto a criação do perfil de usuário e a recomendação propriamente dita. Além disso, deve-se também levar em consideração as dependências do projeto, desde os bibliotecas que depende e os dados necessitados. Caso o projeto necessite de dados que não possam ser obtidos pelo pesquisadores deste trabalho, o mesmo não poderá ser considerado.

    \item \textit{\textbf{Obter código fonte}}
Obter o código fonte do projeto em questão, para que o mesmo possa ser alterado visando o estudo das hipóteses levantadas

    \item \textit{\textbf{Analisar projeto selecionado}}
Nesta etapa, o projeto deve ser analisado, visando a compreensão de como o mesmo funciona e, principalmente, os dados necessários para a execução adequada do projeto.

    \item \textit{\textbf{Coleta dos dados}}
Para prover os possíveis dados necessários para o projeto selecionado, uma estratégia de coleta de dados precisa ser definida e implementada.

    \item \textit{\textbf{Análise dos dados}}
Será necessário analisar os dados coletados, visando descartar dados inválidos e classificar o que foi coletado conforme as necessidades do projeto escolhido.

    \item \textit{\textbf{Executar projeto selecionado}}
Com os dados e as dependências do projeto resolvidas, o mesmo deve ser executado, visando encontrar possíveis problemas de execução que não foram previstos e também observar as funcionalidades do projeto.

    \item \textit{\textbf{Coletar informações temporais}}
Deve-se implementar uma forma de coletar as informações temporais dos pacotes sendo usados pelo usuário e também prover uma forma de classificar essa informação coletada.

    \item \textit{\textbf{Implementar abordagem deterministica}}
Para responder a primeira questão proposta, será necessária implementar uma equação matemática que use as informações de tempo coletadas, juntamente com os outros atributos já usados no projeto selecionado para prover um recomendação.

    \item \textit{\textbf{Implementar abordagem por aprendizado de máquina}}
Implementar uma abordagem não-deterministica, no caso uma estratégia de aprendizado de máquina, para usar as informações de tempo coletadas juntamente com os outros atributos providos pelo projeto selecionado para compor um a recomendação.

    \item \textit{\textbf{Comparar resultados}}
Para responder as hipóteses propostas será necessário não só comporar as duas abordagens propostas, determinística e não determinística, e sim comparar ambas também com as abordagens já implementadas pelo projeto proposto e verificar se as hipótes se sustetam.

    \item \textit{\textbf{Realizar conclusões}}
Com a comparações de resultados feita, a mesma deve ser analisada e conclusões devem ser geradas baseada nos fatos apresentados.
\end{enumerate}

OBS: é necessário validar essas questões tanto com o Luciano quanto com o professor Paulo para que o diagrama final seja implementado.

\section{Teste de hipótese}

O projeto selecionado para testar as hipóteses foi o \textit{AppRecommender}, o recomendador de pacotes para distribuições Debian que implementa tanto estratégias de recomendação por conteúdo quanto colaborativas \cite{araujo2011apprecommender}.
Apesar do AppRecommender usar questões temporarais providas pelo pacote popularity-contest nas suas recomendações, tal atributo temporal só é explorado nas recomendações colaborativas, e não nas baseadas em conteúdo, fazendo com que o projeto passe no critério de seleção estabelecido.

\begin{description}

\item[Obtenção do código fonte]

O código fonte da aplicação AppRecommender pode ser encontrada no repositório de projetos Github \footnote{\url{https://github.com/tassia/AppRecommender}}.
Vale ressaltar que as modificações realizadas foram realizadas em uma cópia local do mesmo, sendo essa mantida pelos pesquisadores deste trabalho.

\end{description}




\subsection{Numeração de Páginas}

A contagem sequencial para a numeração de páginas começa a partir da
primeira folha do trabalho que é a Folha de Rosto, contudo a numeração em
si só deve ser iniciada a partir da primeira folha dos elementos textuais.
Assim, as páginas dos elementos pré-textuais contam, mas não são numeradas
e os números de página aparecem a partir da primeira folha dos elementos
textuais que é a Introdução.

Os números devem estar em algarismos arábicos (fonte Times ou Arial 10) no
canto superior direito da folha, a 02 cm da borda superior, sem traços,
pontos ou parênteses.

A paginação de Apêndices e Anexos deve ser contínua, dando seguimento ao
texto principal.

\subsection{Espaços e alinhamento}

Para a monografia de TCC 01 e 02 o espaço entrelinhas do corpo do texto
deve ser de 1,5 cm, exceto RESUMO, CITAÇÔES de mais de três linhas, NOTAS
de rodapé, LEGENDAS e REFERÊNCIAS que devem possuir espaçamento simples.
Ainda, ao se iniciar a primeira linha de cada novo parágrafo se deve
tabular a distância de 1,25 cm da margem esquerda.

Quanto aos títulos das seções primárias da monografia, estes devem começar
na parte superior da folha e separados do texto que o sucede, por um espaço
de 1,5 cm entrelinhas, assim como os títulos das seções secundárias,
terciárias.

A formatação de alinhamento deve ser justificado, de modo que o texto fique
alinhado uniformemente ao longo das margens esquerda e direita, exceto para
CITAÇÕES de mais de três linhas que devem ser alinhadas a 04 cm da margem
esquerda e REFERÊNCIAS que são alinhadas somente à margem esquerda do texto
diferenciando cada referência.

\subsection{Quebra de Capítulos e Aproveitamento de Páginas}

Cada seção ou capítulo deverá começar numa nova pagina (recomenda-se que
para texto muito longos o autor divida seu documento em mais de um arquivo
eletrônico).

Caso a última pagina de um capitulo tenha apenas um número reduzido de
linhas (digamos 2 ou 3), verificar a possibilidade de modificar o texto
(sem prejuízo do conteúdo e obedecendo as normas aqui colocadas) para
evitar a ocorrência de uma página pouco aproveitada.

Ainda com respeito ao preenchimento das páginas, este deve ser otimizado,
evitando-se espaços vazios desnecessários.

Caso as dimensões de uma figura ou tabela impeçam que a mesma seja
posicionada ao final de uma página, o deslocamento para a página seguinte
não deve acarretar um vazio na pagina anterior. Para evitar tal ocorrência,
deve-se re-posicionar os blocos de texto para o preenchimento de vazios.

Tabelas e figuras devem, sempre que possível, utilizar o espaço disponível
da página evitando-se a \lq\lq quebra\rq\rq\ da figura ou tabela.

\section{Coleta de dados}

A coleta de dados é o método no qual usuários do AppRecommender poderão
colaborar com a melhoria do sistema, pois esses dados serão utilizados para
análise de eficiência da recomendação para um perfil de usuário.

Afim de obter dados do usuário tanto quanto aos pacotes que esse utiliza,
suas preferências e perfil, serão coletados certos dados, sendo eles:

\begin{itemize}
    \item \textit{\textbf{Pacotes instalados:}} Necessário para se ter conhecimento dos pacotes que o usuário utiliza;
    \item \textit{\textbf{Pacotes manualmente instalados:}} Estes serão coletados para filtrar as preferências do usuário, para que quando comparado aos pacotes instalados se possa diferenciar pacotes que são automaticamente instalados dos que foram manualmente instalados;
    \item \textit{\textbf{Tempo de acesso e de modificação dos pacotes:}} Estes dados serão usados afim de obter informação quanto a classificação do pacote em relação ao tempo que este foi utilizado;
    \item \textit{\textbf{Caminho do binário de cada pacote:}} Dados coletados afim de se saber qual o binário responsável pela execução do pacote;
    \item \textit{\textbf{Versão do Kernel:}} Necessário para agrupar as recomendações e preferências entre usuários com mesmo kernel;
    \item \textit{\textbf{Distribuição do sistema:}} Necessário para agrupar as recomendações e preferências entre usuários com a mesma distribuição.
    \item \textit{\textbf{Dados do popcon:}} Informação que é utilizada para permitir executar os experimentos afim de executar os experimentos de análise dos dados.
\end{itemize}

A qualificação da recomendação também será analisada pela coleta de dados
quanto a qual recomendação tem uma maior relevância para o usuário, onde
uma recomendação é o conjunto de pacotes resultantes da execução do
AppRecommender com certos parâmetros definidos. Segue o processo para
coleta da relevância da recomendação para o usuário:

\begin{itemize}
    \item \textit{\textbf{Execução do AppRecommender:}} O AppRecommender é executado várias vezes com parâmetros diferentes, onde cada execução resulta em uma recomendação diferente;
    \item \textit{\textbf{Agrupamento das recomendações:}} Todas as recomendações são agrupadas, ordenadas e são removidas as duplicações de pacotes entre as recomendações;
    \item \textit{\textbf{Pontuação do usuário:}} Para cada pacote no agrupamento das recomendações é solicitado ao usuário que seja dada uma nota de 0 a 10 ao pacote;
    \item \textit{\textbf{Armazenamento das informações:}} É coletado, ou seja, armazenado, cada uma das recomendações, o agrupamento das recomendações e as pontuações de cada pacote.
\end{itemize}

Para armazenar a coleta de dados é criado um diretório na raíz de diretórios
do usuário, onde os dados coletados são
armazenados em arquivos de texto, onde cada arquivo possui o nome de acordo
com o dado que foi coletado e armazenado.

\section{Cópias}

Nas versões do relatório para revisão da Banca Examinadora em TCC1 e TCC2,
o aluno deve apresentar na Secretaria da FGA, uma cópia para cada membro da
Banca Examinadora.

Após a aprovação em TCC2, o aluno deverá obrigatoriamente apresentar a
versão final de seu trabalho à Secretaria da FGA na seguinte forma:

\begin{description}
	\item 01 cópia encadernada para arquivo na FGA;
	\item 01 cópia não encadernada (folhas avulsas) para arquivo na FGA;
	\item 01 cópia em CD de todos os arquivos empregados no trabalho;
\end{description}

A cópia em CD deve conter, além do texto, todos os arquivos dos quais se
originaram os gráficos (excel, etc.) e figuras (jpg, bmp, gif, etc.)
contidos no trabalho. Caso o trabalho tenha gerado códigos fontes e
arquivos para aplicações especificas (programas em Fortran, C, Matlab,
etc.) estes deverão também ser gravados em CD.

O autor deverá certificar a não ocorrência de “vírus” no CD entregue a
secretaria.

