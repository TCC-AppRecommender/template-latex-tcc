\begin{resumo}

Sistemas de recomendação estão ficando cada vez mais populares. Diversas
aplicações que contém uma grande gama de itens ou serviços para prover aos seus
usuários já estão usando sistemas de recomendação para ajudar usuários na
tarefa de melhor acharem suas preferências no sistema, como o Netflix e a Amazon.
Um exemplo onde tal sistema pode ser empregado é em distribuiçõs GNU/Linux, como
o Debian, onde o número de pacotes disponíveis ao usuário cresce diariamente,
contendo atualmente mais de 49000 pacotes disponívies para seus usuários.
Dessa forma, recomendar o pacote certo para o usuário pode ajudar não só o mesmo
na realização de suas atividades, como contribuir com a comunidade como um todo.
Esta pesquisa visa então propor o uso de contexto temporal de uso dos pacotes do
usuário como diferencial na criação do seu perfil de recomendação. com a
hipótese de que tal informação irá prover melhores recomendações. Para isso,
essa pesquisa irá usar como base o software AppRecommender,
que já provê recomendação de pacotes para usuários GNU/Linux.


 \vspace{\onelineskip}
    
 \noindent
 \textbf{Palavras-chaves}: Sistemas de recomendação, Debian, Aprendizado de
 máquina.
\end{resumo}
